\section{Назаренко Вячеслав - Индивидуальный Инвестиционный Счёт}
Бюджет: 100.000 евро \\


\subsection{Что такое ИИС?}

\textit{Индивидуальный Инвестиционный Счёт} - это особый брокерский счёт или счёт доверительного управления физического лица, доступный к открытию в России с 1 января 2015 года. По заявлению правительства данная инициатива направлена на стимуляцию инвестиций населения в ценные бумаги государственных компаний.

Разберём возможности и ограничения такого счёта.\\
Ограничения:
\begin{enumerate}
    \item В течение одного календарного года внести на счет можно не более 1 млн.руб.
    \item Средства на счёте можно использовать только для покупки ценных бумаг, обращающихся на российской бирже.
\end{enumerate}
Возможности:
\begin{enumerate}
    \item Физическое лицо, открывшее счёт получает право на \textit{налоговый вычет} одного из двух типов по выбору.
\end{enumerate}

Налоговый вычет - возврат  физическому лицу части уплаченного им НДФЛ. Существуют два типа НВ:
\begin{enumerate}
    \item (\textbf{Тип А}) Возврат части уплаченного физ. лицом НДФЛ по основному месту работы за прошедший календарный год, но не более минимума из трех величин:
    \begin{enumerate}
        \item 13\% от суммы пополнения счета за прошедший календарный год
        \item 52000 руб.
        \item Величина уплаченного за прошедший календарный год НФДЛ по основному месту работы
    \end{enumerate}
    \item (\textbf{Тип Б}) Полный возврат удержанного налога от дохода по ИИС при его закрытии.
\end{enumerate}
НВ выплачивается только в том случае, если данный ИИС просуществовал хотя бы три года. Рассмотрим подробнее налогообложение акций и облигаций.
\subsection{Налогообложение акций и облигаций}

\textbf{Облигации.}
\begin{itemize}
    \item Проценты по муниципальным облигациям и ОФЗ (Облигации Федерального Займа) налогом \textbf{не облагаются.}
    \item Облигации, чья купонная ставка не превышает 5\% + (Ключевая Ставка ЦБ РФ (7.5\% на 18 ноября 2018 года)) налогом \textbf{не облагаются.}
    \item С остальных облигаций удерживается налог, вычисляемый по следующей формуле: 
    $$Taxes = NominalBondValue\ \cdot\ (CouponRate - 5\% - KeyRate) \cdot\ 35\%$$
\end{itemize}
\textbf{Акции.} Доход с дивидентов облагается налогом размером в 9\%.\\
\textbf{Купля-продажа.} Доход с купли-продажи уплачивается в размере 13\%.

\subsection{Исходные данные}
\subsubsection{Бюджет}
Дальнейшие вычисления ожидаемой прибыли  будут проводиться с \textbf{исходным бюджетом равным 100 000 EU.} 

Также считается, что есть 6 человек, готовых открыть на свое имя ИИС. Каждый из них официально трудоустроен и каждый год уплачивает НДФЛ в размере превышающем 52000 руб.
\subsubsection{Конвертация валюты.}
Первым шагом к осуществлению инвестирования с помощью ИИС должна быть конвертация бюджета в российскую валюту. Согласно ресурсу \textit{sravni.ru} наиболее выгодный курс продажи евро на 18 ноября 2018 года предлагает Банк Интеза (72.6 руб). По информации от оператора горячей линии банка, для осуществления обмена такой суммы необходимо стать клиентом банка, заранее заказать привоз необходимых для обмена денежных средств в отделение банка и осуществить продажу евро. Никаких дополнительных расходов клиент не несет.

Но для простоты вычислений мы обменяем по курсу 72 рубля за 1 евро.
\nopagebreak Итак, после конвертации у нас на руках \textbf{7 200 000 руб.} на \textbf{6 человек.}
\subsection{Выбор брокера}
В России услуги по открытию и обслуживанию ИИС предоставляют множество банков и брокеров. Так выглядела ситуация по количеству открытых ИИС на 1 января 2017 года:
\insertpicc{slava/iis-dec-2016-allbrokers-1-1.jpg}{Открытые счета за 2016 год}

Мы сравним условия, предлагаемые по ИИС в следующих компаниях:
\begin{enumerate}
    \item Сбербанк
    \item Финам
    \item Открытие
    \item БКС
    \item Альфа-капитал
    \item ВТБ
\end{enumerate}

\subsubsection{Сбербанк}
Сбербанк предлагает два тарифа (не путать с типом НВ): ``Самостоятельный''и ``Инвестиционный''. 
\insertpdf{slava/Sberbank_-_Tarify.pdf}

Инвестиционный тариф - тариф доверительного управления. Сразу отметим, что для получения наибольшей относительной выгоды от налогового вычета первого типа, каждый год на ИИС будет вноситься 400 тыс. руб. Соответственно сделок мы можем максимум заключить на те же деньги. В таком случае комиссия в размере 0.125 \% превышает среднюю по нашей выборке минимум в два раза, что неприемлимо. В связи с этим дальнейшее рассмотрение брокера не проводится.

\subsubsection{Финам}
Финам предлагает следующие \textbf{варианты управления ИИС:}
\begin{enumerate}
    \item Обычный ИИС, средствами которого можно управлять через приложение FinamTrade
    \item ``Автоследование'' с помощью сервиса comon.ru. Наш ИИС синхронизируется с брокерским счетом управляющего по нашему выбору и все его сделки проводятся и с нашими средствами в отношении $\frac{OurIIAFunds}{ManagerAccountFunds}$.
    \item ``Доверительное управление''. Управляющий распоряжается нашим счетом по своему усмотрению. Минимальная сумма 500 000 руб.
\end{enumerate}
\textbf{Комиссия брокера.}

 За операции на фондовом рынке брокером взимается комиссия 0.0354\%. 

В случае Автоследования управляющий взимает вознаграждение от 1\% до 15\% годовых от Суммы Чистых Активов (управляющий сам выбирает тарификацию). Также есть ежемесячная абонентская плата за услугу от 10 руб до 5000 руб ( также по выбору управляющего). 

В случае Доверительного Управления комиссия составит 2 \% годовых от СЧА и 15 \% от инвестиционного дохода

\subsubsection{Открытие}
Открытие предлагает следующие тарифы
\begin{itemize}
    \item Самостоятельное управление
    \item Модельный портфель ``Доходный''
    \item Модельный портфель ``Сберегательный''
    \item Модельный портфель ``Независимый''
\end{itemize}

Самостоятельное управление предлагается как через мобильное приложение, так и через терминал QUIK. Работа с портфелем бесплатная. Модельные портфели все так же представляют собой самостоятельное управление, но с рекомендациями от управляющей компании. Рекомендации отображаются в личном кабинете, следование им осуществляется по нажатию одной кнопки и является опциональным. Портфели отличаются между собой основными приобретаемыми бумагами и агрессивностью стратегий.

\textbf{Комиссия брокера.}

За операции на фондовом рынке брокером взимается комиссия
\begin{itemize}
    \item Самостоятельное управление: 0.057\%
    \item Модельные портфели: 0.2\%
\end{itemize}


\subsubsection{БКС}
БКС предлагает один единственный тариф ``Инвестиционный'' c самостоятельным управлением через свое мобильное приложение.

\insertpicc{slava/bks-tarif.png}{Выжимка о тарифе ИИС}


\subsubsection{Альфа-капитал}

Альфа-капитал предлагает только ИИС с доверительным управлением. Существуют три стратегии, отличающиеся по агрессивности и составу портфеля.

\begin{itemize}
    \item Точки роста
    \item Наше будущее
    \item Новые горизонты
\end{itemize}

\insertpicc{slava/alpha-tarif.png}{Информация с официального сайта Альфа-капитал}

Заметим, что доходность в процентах  у ``Новых горизонтов'' больше, чем у ``Нашего будущего'', но красные кружочки говорят об обратном.

\textbf{Комиссия брокера.}

Со стратегией ``Наше будущее'' Альфа-капитал взимает 1\% при внесении средств и ежеквартально 0.375\% от СЧА. С остальными двумя стратегиям взимается одинаковая комиссия по той же схеме, но соответствующие проценты уже равны 2\% и 0.5\%

\newpage

\subsubsection{ВТБ}
Для не привилегированных клиентов доступны два базовых тарифа с самостоятельным управлением. Управление может осуществляться как с помощью мобильного приложения, так и с помощью терминала QUIK.

\begin{itemize}
    \item Инвестор стандарт
    \item Профессиональный стандарт
\end{itemize}

\textbf{Комиссия брокера.}
Ежемесечная стоимость обслуживания счета 150 руб.
\insertpicc{slava/vtb-tarif.png}{Информация с официального сайта ВТБ}

\newpage
\subsection{Сравнение брокеров}

Проведем сравнение тарифов с самостоятельным управлением без рекомендаций (отсутствует у Альфа-капитала).
\\
\begin{tabular}{|c|c|c|c|c|c|}
    \hline
    & Сбербанк & Финам & Открытие & БКС & ВТБ \\
    \hline
    Комиссия от оборота*, \% & 0.125 & 0.0354 & 0.057 & 0.0531 & 0.0413 \\
    \hline
    Мин. комиссия за сделку, руб & 0 & 41.3 & 0.04 & 0    & 0 \\
    \hline
    Мин. комиссия за день, руб   & 0 & 0    & 0    & 35.4 & 0 \\
    \hline
    Ежемесячная плата, руб       & 149 & $\geq$ 177** & 0*** & 177 или 0**** & 150 \\
    \hline
\end{tabular}
\begin{footnotesize}{* При сделках в день в объеме 80000 руб.}\end{footnotesize}\\
\begin{footnotesize}{** Депозитарная плата (если были сделки в текущем месяце) + Абонентская - уплаченная комиссия}\end{footnotesize}\\
\begin{footnotesize}{*** Так как на нашем счету СЧА будет больше 50000}\end{footnotesize}\\
\begin{footnotesize}{**** Взимается если были сделки в текущем месяце}\end{footnotesize}
\\

Каждый брокер предоставляет возможность управлять счётом через мобильное приложение.

Мы не будем сравнивать доверительное управление, так как наша задача сохранить деньги с бОльшим процентом, чем в банковских вкладах. Для этой цели мы будем покупать и держать облигации и пользоваться налоговым вычетом.

\subsection{Стратегия вложений}

\subsubsection{Налоговый вычет}
Для максимизации относительной прибыли мы хотим получать налоговый вычет наибольшее количество раз в наибольшем размере.\\
$AccountsCount = 6$\\
$TotalFunds = 7 200 000$\\
$TaxDeductionsCount = \left[\displaystyle{\frac{7 200 000}{400 000}}\right] = 18$\\
Для того, чтобы каждый год реинвестировать в облигации максимальное количество денег, имеет смысл вкладывать ежегодно по 400000 руб. в каждый из 6 счетов. Тогда для реинвестирования в следующем году будет доступно\\
$AnnualTaxDeductions = 52 000 \times AccountsCount = 312 000$\\
\\
$YearsOfTaxDeductions = \displaystyle{\frac{TaxDeductionsCount}{AccountsCount}} = 3$
\subsubsection{Выбор облигаций}
Облигации будем рассматривать только тех эмитентов, чей рейтинг согласно хотя бы одному рейтинговому агентству из Большой Тройки  не ниже B+ и чей срок погашения не выходит за временные рамки нашего эксперимента.

Доступные на данный момент облигации c высоким процентом:

\begin{tabular}{|c|c|c|}
    \hline
    Облигация & Проценты годовых, \%& Рейтинг*\\
    \hline
    ПКТ 03    & 12.5                &  BB     \\
    \hline
    Полюс Б1  & 12.1                &  BB-    \\
    \hline
    СПбТел 02   & 11.3              &  B+     \\
    \hline
    Металин БО2 & 10.95             &  BB     \\
    \hline
    Система1р1  & 9.75              &  BB-    \\
    \hline
\end{tabular}\\
\begin{footnotesize}
    * Наибольший из присвоенных одним из рейтинговых агентств Большой Тройки
\end{footnotesize}\\

Для простоты вычислений будем считать, что каждый год мы можем вложить свободные деньги в 5 различных облигаций со средним процентом годовых
$p = \displaystyle{\frac{12.5 + 12.1 + 11.3 + 10.95 + 9.75}{5}} = 11.32$.

Тогда приблизительный объем средств на конец эксперимента можно рассчитать так:
$AnnualInvestments = 400 000 \times AccountsCount = 2 400 000$\\
$InvestmentsCount = \left[\displaystyle{\frac{TotalFunds}{AnnualInvestments}}\right] = 3$\\
$TotalEndFunds = (((((AnnualInvestments - Taxes) \times 1.1132 + AnnualTaxDeductions + AnnualInvestments - Taxes) \times 1.1132 + AnnualTaxDeductions + AnnualInvestments - Taxes) \times 1.1132 +
AnnualTaxDeduction - Taxes) \times 1.1132 - Taxes) \times 1.1132 + (TotalFunds - AnnualInvestments \times 3) \times 1.1132^5$ 

\subsubsection{Подсчет прибыли}
С помощью программы на python, симулирующий нашу стратегию рассчитаем приблизительную прибыль на одном счете с каждым брокером.
\newpage
\textbf{Сбербанк.}\\
\begin{tabular}{|c|c|c|c|c|c|}
    \hline
    Год & В облигациях & Чистые вложения & Прибыль & Налоговый вычет & Комиссия\\
    \hline
    Первый год & 397712.00 & 400000 & 45021.00 & 52000 & 2288.00\\
    \hline
    Второй год & 494611.72 & 400000 & 101011.05 & 52000 & 2409.28\\
    \hline
    Третий год & 550531.78 & 400000 & 163331.24 & 52000 & 2479.26\\
    \hline
    Четвертый год & 213187.95 & 0 & 187464.12 & 0 & 2143.30\\
    \hline
    Пятый год & 185366.80 & 0 & 208447.64 & 0 & 2097.32\\
    \hline
\end{tabular}
\\
Суммарная прибыль: 861275.05\\
\\
\\
\textbf{Финам.}\\
\begin{tabular}{|c|c|c|c|c|c|}
    \hline
    Год & В облигациях & Чистые вложения & Прибыль & Налоговый вычет & Комиссия\\
    \hline
    Первый год & 397669.50 & 400000 & 45016.19 & 52000 & 2330.50\\
    \hline
    Второй год & 494685.69 & 400000 & 101014.61 & 52000 & 2330.50\\
    \hline
    Третий год & 550684.11 & 400000 & 163352.05 & 52000 & 2330.50\\
    \hline
    Четвертый год & 213021.55 & 0 & 187466.09 & 0 & 2330.50\\
    \hline
    Пятый год & 185135.59 & 0 & 208423.44 & 0 & 2330.50\\
    \hline
\end{tabular}
\\
Суммарная прибыль: 861272.37\\
\\
\\
\textbf{Открытие.}\\
\begin{tabular}{|c|c|c|c|c|c|}
    \hline
    Год & В облигациях & Чистые вложения & Прибыль & Налоговый вычет & Комиссия\\
    \hline
    Первый год & 399772.00 & 400000 & 45254.19 & 52000 & 228.00\\
    \hline
    Второй год & 496970.76 & 400000 & 101511.28 & 52000 & 283.43\\
    \hline
    Третий год & 553195.78 & 400000 & 164133.04 & 52000 & 315.50\\
    \hline
    Четвертый год & 216009.85 & 0 & 188585.36 & 0 & 123.20\\
    \hline
    Пятый год & 188477.86 & 0 & 209921.05 & 0 & 107.49\\
    \hline
\end{tabular}
\\
Суммарная прибыль: 865404.92\\
\\
\\
\textbf{БКС.}\\
\begin{tabular}{|c|c|c|c|c|c|}
    \hline
    Год & В облигациях & Чистые вложения & Прибыль & Налоговый вычет & Комиссия\\
    \hline
    Первый год & 399787.60 & 400000 & 45255.96 & 52000 & 212.40\\
    \hline
    Второй год & 496991.91 & 400000 & 101515.44 & 52000 & 264.04\\
    \hline
    Третий год & 553276.88 & 400000 & 164146.38 & 52000 & 238.57\\
    \hline
    Четвертый год & 215969.38 & 0 & 188594.12 & 0 & 177.00\\
    \hline
    Пятый год & 188417.12 & 0 & 209922.94 & 0 & 177.00\\
    \hline
\end{tabular}
\\
Суммарная прибыль: 865434.83\\
\\
\\
\textbf{ВТБ.}\\
\begin{tabular}{|c|c|c|c|c|c|}
    \hline
    Год & В облигациях & Чистые вложения & Прибыль & Налоговый вычет & Комиссия\\
    \hline
    Первый год & 398034.80 & 400000 & 45057.54 & 52000 & 1965.20\\
    \hline
    Второй год & 495052.25 & 400000 & 101097.45 & 52000 & 2005.28\\
    \hline
    Третий год & 551069.03 & 400000 & 163478.47 & 52000 & 2028.43\\
    \hline
    Четвертый год & 213589.48 & 0 & 187656.80 & 0 & 1888.99\\
    \hline
    Пятый год & 185779.29 & 0 & 208687.01 & 0 & 1877.50\\
    \hline
\end{tabular}
\\
Суммарная прибыль: 861977.27\\
\\
\\

Наибольшую прибыль обещает брокер БКС.\\
Относительная прибыль $= \displaystyle{\frac{865434.83}{1200000}} \times 100 \%  = 72.12 \%$\\
Заметим, что это равносильно $(1.7212^\frac{1}{5} - 1) \times 100\% = 11.472\%$ годовых. \\

\textbf{Почему так мало?} Этот процент едва выше предполагаемых годовых ($11.32$) от облигаций без учета налогового вычета. Это объясняется тем, что деньги вкладываются постепенно, а не сразу. И это приводит нас к интересному выводу. Но прежде чем этот вывод озвучить, подведем итоги по сохранению наших вложений.

\textbf{Итоговый объем денежных средств в рублях } составит $7 200 000  + 865434.83 \times 6 = 12\ 392\ 608.98$

\subsection{Итог}
Как видно, ограничения по максимальному ежегодному налоговому вычету и наше желание использовать вложенные средства для получения наибольшего возможного налогового вычета практических нивелировали его процентный вклад в итоговую сумму. Из этого следует вывод, что обладая средствами для ежегодных инвестиций в размере 400000 руб и желая использовать стратегию ``купи и держи'',
\begin{center}
\textbf{выбирать ИИС вместо обычного  брокерского счета нецелесообразно.}
\end{center}